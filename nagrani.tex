\documentclass[reprint,amsmath,amssymb,aps,twoside]{revtex4-2}

\usepackage{graphicx}
\usepackage{amsmath,amssymb,amsfonts}
\usepackage{dcolumn}
\usepackage{bm}
\usepackage{siunitx}
%\usepackage{tikz,pgfplots}
\sisetup{separate-uncertainty=true, multi-part-units=single}
\usepackage[colorlinks,allcolors=blue]{hyperref}
\usepackage{cleveref}
\crefname{equation}{}{}
\crefname{figure}{Fig.}{Figs.}
\crefname{table}{Table}{Tables}
\usepackage{svg}
\svgpath{{./data/}}

% set PDF metadata
\hypersetup{%
pdftitle={Gravitational acceleration affects falling objects equally},
pdfauthor={Sejal Nagrani, Julia Bawar, and Kelly Su},
}
\usepackage{fancyhdr}
\pagestyle{fancy}
\fancyhf{}
\fancyhead[RE,RO]{J S\&E \textbf{2}, 29--31 (2026)}
\fancyhead[LO]{Nagrani, Bawar, and Su}
\fancyhead[LE]{Gravitational acceleration affects falling objects equally}
\fancyfoot[C]{\thepage}
\fancypagestyle{mytitlepage}{
\fancyhf{}
\fancyhead[C]{Journal of Science \& Engineering \textbf{2}, 29--31 (2026)}
\fancyfoot[C]{\thepage}
}


\begin{document}
\setcounter{page}{29}

\title{Gravitational acceleration affects falling objects equally}
\author{Sejal Nagrani}
\email{Contact author: 427snagrani@frhsd.com}
\author{Julia Bawar}
\author{Kelly Su}
\altaffiliation{\href{https://manalapan.frhsd.com/}{Manalapan High School}, Englishtown, NJ 07726 USA}
\affiliation{Science \& Engineering Magnet Program, \href{https://manalapan.frhsd.com/}{Manalapan High School}, Englishtown, NJ 07726 USA}
\date{\today}

\begin{abstract}
This experiment examines two hypotheses about the motion of falling objects: Aristotle’s idea that heavier objects fall faster than lighter ones, and Galileo’s idea that objects fall at a constant acceleration when air resistance is negligible. Five drop tests were conducted in which we released a kickball ($m=\qty{0.210}{\kilo\gram}$) and a baseball ($m=\qty{0.144}{\kilo\gram}$) from a height of \qty{5}{\meter} to obtain the times it took for each object to fall a distance of \qty{5}{\meter}. The measurements allowed us to estimate the acceleration experienced by each. Baseballs fell in \qty{0.84\pm0.08}{\second}, while kickballs fell in \qty{0.84\pm0.07}{\second}; the resulting accelerations were \qty{14.7\pm2.8}{\meter\per\second\squared} and \qty{14.5\pm2.4}{\meter\per\second\squared}, respectively. Accelerations were approximately equal, regardless of mass; and differences between time to fall and between acceleration were not statistically significant, thus supporting Galileo’s hypothesis that objects fall at a constant acceleration when air resistance is negligible.
\end{abstract}

%\keywords{keywords here}

\maketitle\thispagestyle{mytitlepage}



\section{Introduction}
In one-dimensional kinematics, the position function $y(t)$ is a function that connects time $t$, acceleration $a$, initial velocity $v_0$, and initial position $y_0$ to produce a final position of an object\cite{tipler, openstax, barrons}:
\begin{equation}
y = \frac{1}{2} a t^2 + v_0 t + y_0.
\label{eq:1}
\end{equation}
We have rewritten \cref{eq:1} to solve for the gravitational acceleration constant $a = -g$, assuming that the initial velocity $v_0 = \qty{0}{\meter\per\second}$, initial position $y_0 = h$, and final position is $y=0$ at $t$:
\begin{equation}
g = \dfrac{2 h}{t^2}.
\label{eq:2}
\end{equation}

The importance of gravity has long been recognized in physics. The gravitational constant $g = \qty{9.81}{\meter\per\second\squared}$ describes the acceleration of an object falling near the Earth's surface \cite{tipler, openstax, barrons}; but this has not always been known. In antiquity, Aristotle asserted that heavier objects fall faster than lighter objects \cite{aristotle:physics}, an assertion that was not challenged until Galileo Galilei, an Italian physicist, proposed that all objects fall at the same rate of acceleration when air resistance is negligible \cite{galilei:1638:discorsi, hall:2022:motion, machamer:2021:galileo}. To test their hypotheses, we predicted that the acceleration of falling objects is constant regardless of their mass, and then conducted several drop tests using a kickball and a baseball. 

The null hypothesis and alternative hypothesis are listed below: 
\begin{align}
H_0: \bar{t}_{k} &= \bar{t}_{b} \\
H_1: \bar{t}_{k} &< \bar{t}_{b}
\end{align}
where $\bar{t}_{k}$ and $\bar{t}_{b}$ denote the mean fall times of a kickball and baseball, respectively.  The null hypothesis $H_0$ states that there is no difference in the mean fall time of the two objects, after Galileo \cite{galilei:1638:discorsi, machamer:2021:galileo}, while the alternative hypothesis $H_1$ states that the heavier object (kickball) falls faster, after Aristotle \cite{aristotle:physics}. 

We could test among these hypotheses simply by dropping both objects and seeing which one hits the ground first. However, we chose to estimate the gravitational accelerations experienced by each as a further test. To solve for the gravitational acceleration $g$, we measured the fall time $t$ and used \cref{eq:2}. This allowed us to compare the objects’ respective accelerations to see if they were equal. To verify our hypothesis, we also ran a two-sample $t$-test, assuming unequal variances, to further ensure its accuracy \cite{starnes:2015:practice}.




\begin{figure}[b]
\begin{center}
\includegraphics{data/newfig1.pdf}
\end{center}
\caption{A. Outdoor experimental setup depicting the drop height (\qty{5}{\meter}) used for timing falling objects. B. (left) Kickball ($m=\qty{0.210}{\kilo\gram}$) and (right) baseball ($m=\qty{0.144}{\kilo\gram}$).} 
\label{fig:1}
\end{figure}

\section{Methods and materials}

\subsection{Drop tests}
In our experiment, we had one person stand inside a classroom at a second-story window, approximately \qty{5}{\meter} above the ground, to release objects, as shown in \cref{fig:1}A. Two of those objects were a kickball ($m=\qty{0.210}{\kilo\gram}$; Walmart; Freehold, NJ) with a circumference of \qty{0.565}{\meter} and a baseball ($m=\qty{0.144}{\kilo\gram}$; Walmart; Freehold, NJ) with a circumference of \qty{0.23}{\meter}, shown in \cref{fig:1}B, which we focused on for the remainder of the experiment. 

A group of observers outside at ground level measured time to fall, where each person managed one digital stopwatch (Pulivia YS-802; Shenzhen, China) with a precision of \qty{\pm 0.01}{\second}. Additional observers outside at ground level filmed each trial at 30 and 60 \unit{frame\per\second} using a smartphone (iPhone 13; Apple Inc; Cupertino, CA).  

For each trial, a sequential countdown from three down to one and a verbal command ``drop'' were communicated via megaphone (Pyle USA; Brooklyn, NY) to signal the person at the window to release the object from his hands by removing his hands from contact with the object or by opening his fist, depending on the size of the object. On ``drop'', the timers would begin their stopwatches and, through visual observation, would stop their stopwatches when the object made contact with the ground. The videographers would begin recording before the countdown and stop recording a couple of seconds after the object hits the ground. The times obtained by each stopwatch were then recorded for statistical analysis. This process was repeated over five trials, one object per trial, where each object was released from the same person's hands at the same height to maintain consistency. The experiment was conducted outdoors under calm weather conditions to minimize the effect of wind and other external factors on the objects as they fell.

%Five trials were sufficient to estimate the mean fall time and variability while minimizing the effects of human reaction time error because the experimental conditions were held constant across all trials. Thus, additional trials were not expected to make any significant change in the measured mean fall times, $\bar{x}_{k}$  and $\bar{x}_{b}$. 

\subsection{Analysis}
Statistical analysis \cite{starnes:2015:practice} was done in R \cite{R} using the \texttt{dplyr} and \texttt{ggplot2} packages \cite{dplyr, ggplot2}. For each measured time, \cref{eq:2} was used to estimate $g$. $t$-tests were then performed on measured fall times as well as on the estimates of $g$ from each drop. Data and analysis code are available at \url{https://github.com/devangel77b/427snagrani-lab1.git}








\section{Results}
\input{tables/table2.tex}
\begin{table}[t]
\caption{Fall times for kickball and baseball (mean $\pm$ 1 s.d.), for $n=5$ drops each, and corresponding estimates of $g$ based on \cref{eq:2}. Differences are not significant between baseball and kickball ($t$-test, $p=0.9269$ for $t$, $p=0.8643$ for $g$ estimates).}
\label{tab:4}
\begin{center}
\begin{ruledtabular}
\begin{tabular}{lcc}
type & $t$, \unit{\second} & $g$, \unit{\meter\per\second\squared} \\
\colrule
baseball & \num{0.84\pm0.08} & \num{14.7\pm2.8} \\
kickball & \num{0.84\pm0.07} & \num{14.5\pm2.4} \\
\end{tabular}
\end{ruledtabular}
\end{center}
\end{table}

%\begin{table}[t]
%\caption{Fall times for kickball and baseball (mean $\pm$ 1 s.d.), for $n=5$ drops each, and corresponding estimates of $g$ based on \cref{eq:2}.}
%\label{tab:4}
%\begin{center}
%\begin{ruledtabular}
%\begin{tabular}{lcc}
%type & $t$, \unit{\second} & $g$, \unit{\meter\per\second\squared} \\
%\colrule
%baseball & \num{0.84\pm0.08} & \num{14.7\pm2.8} \\
%kickball & \num{0.84\pm0.07} & \num{14.5\pm2.4} \\
%\end{tabular}
%\end{ruledtabular}
%\end{center}
%\end{table}
\begin{figure}[b]
\begin{center}
\includesvg{newfig2.svg}
\end{center}
\caption{(left) Bar chart of measured fall times. (right) Bar chart of $g$ estimates. Differences are not significant between baseball and kickball ($t$-test, $p=0.9269$ for $t$, $p=0.8643$ for $g$ estimates).}
\label{fig:2}
\end{figure}

Measured fall times are shown in \cref{tab:2}. \Cref{tab:4} summarizes the measured fall times and estimates of $g$ for kickball and baseball. The data are plotted in \cref{fig:2}. Differences are not significant between baseball and kickball (two-sample $t$-test, $p=0.9269$ for $t$, $p=0.8643$ for $g$ estimates).




\section{Discussion}

\subsection{Aristotle or Galileo?}
As shown in \cref{tab:4} and \cref{fig:2}, the gravitational constant $g_k$ for $n=5$ kickballs was \qty{14.5\pm2.4}{\meter\per\second} (mean $\pm$ 1 s.d.). The gravitational constant $g_b$ for $n=5$ baseballs was \qty{14.7\pm2.8}{\meter\per\second}. Both objects fell with the same acceleration (one sided, two-sample $t$-test, $t=0.17343$, $df=17.529$, $p=0.8643$, $\alpha=0.05$). This result is consistent with Galileo’s hypothesis ($H_0$) that objects fall at the same rate of acceleration when air resistance is negligible \cite{galilei:1638:discorsi}; and we reject Aristotle's alternative $H_1$, that heavier objects fall faster. 

This experiment provides experimental confirmation of Galileo’s conclusion that objects of different masses fall with the same acceleration when air resistance is negligible. Although limited by human reaction time and a small sample size, our results align with the established physical theory and demonstrate the value of statistical analysis in experimental physics.




\subsection{Sources of experimental error}
Our measured values of $g$ (\cref{tab:4} and \cref{fig:2}) are somewhat higher than a typical value of $g=\qty{9.8}{\meter\per\second\squared}$ \cite{tipler,openstax,barrons}. Rearranging \cref{eq:1} gives the time $t$ for an object to fall distance $h$ from rest:  
\begin{equation}
t = \sqrt{\frac{2h}{g}}.
\label{eq:3}
\end{equation}
\Cref{eq:3} predicts the theoretical fall time from a height of \qty{5}{\meter} to be approximately \qty{1.01}{\second}. Our experimentally measured mean fall times (\cref{tab:2,tab:4,fig:2}) were slightly shorter than this value, which we attribute to reaction time error when using handheld stopwatches \cite{hetzler:2008:reliability,faux:2019:manual}, either due to timers starting the stopwatches late or ending them early in anticipation of objects hitting the ground. In hindsight, it may have been beneficial to have the timers give the countdown and verbal command to drop the object.

Although air resistance was neglected, it may still have had a minor effect due to differences in surface area and shape between the baseball and the kickball. Another limitation is inconsistency in drop height, which could have been reduced by providing a landmark at the window. Future experiments could improve precision by using high-speed cameras to eliminate human timing errors, having better communication among experimenters, or automating drop and timing.







\section{Acknowledgments}
We thank our classmates who assisted with data collection and several anonymous reviewers who helped with revisions. SN collected data and documented our method and materials, results, and discussion. JB assisted in data collection, methods and materials, interpreting results,  and revision of the report. KS collected visual data, described the methods and materials, interpreted results, and revised the report.






\bibliography{lab.bib}
%References
%[1] P.A. Tipler and G. Mosca. Physics for Scientists and Engineers, 5th ed. (W H Freeman and Company, New York, 2004).
%
%
%[2] Galilei, Galileo. Discorsi e dimostrazioni matematiche, intorno à due nuoue scienze attenenti alla mecanica & i movimenti locali. 1638. Internet Archive, https://archive.org/details/discorsiedimostr00gali/mode/2up.
%
%
%[3] Hall, Nancy. “Motion of Free Falling Object.” Glenn Research Center | NASA, NASA, 21 July 2022, www1.grc.nasa.gov/beginners-guide-to-aeronautics/motion-of-free-falling-object/.
%
%
%[4] College Board. Statistics Formula Sheet and Tables 2020. College Board, 2020. AP Central, apcentral.collegeboard.org/media/pdf/statistics-formula-sheet-and-tables-2020.pdf.
%
%
%[5] Hardie, R. P., and R. K. Gaye. “The Internet Classics Archive | Physics by Aristotle.” Classics.mit.edu, classics.mit.edu/Aristotle/physics.4.iv.html.

\end{document}
